\documentclass[11pt,a4paper]{article}
%\documentclass[11pt,a4paper]{scrartcl}
\documentclass[11pt,a4paper,oneside]{book}
\usepackage[british,UKenglish,USenglish,english,american]{babel}
%\usepackage[a4paper, total={16cm, 23cm}]{geometry}
\usepackage[tmargin = 1.25in,bmargin = 1.25in,lmargin = 1in,rmargin = 
1in]{geometry}
\usepackage{tikz}
\usepackage{graphicx}
\usepackage{pgfplots}
\pgfplotsset{width=12cm,compat=1.9}
\usepackage{setspace}
\usepackage{chemmacros}
\usepackage{chemfig}
%\usepackage{ghsystem}
%\usechemmodule{redox}
%\usepackage{chemnum}
%\usepackage{bohr}
%\usepackage{elements}
%\usepackage{endiagram}
%\usepackage{modiagram}
%\usepackage{chemgreek}
%\usepackage{mhchem}
\usepackage{esint}
\usepackage{tabularray}

\usepackage{multirow} % For multirows
\usepackage{amssymb} % For the checkmark
\usepackage{graphicx} % For the rotation
\usepackage{array} % To enlarge the vertical space of the table's contents

\usepackage{makecell}

\usepackage{booktabs}
\usepackage{siunitx}

\usepackage{makeidx}
\usepackage{epstopdf}

\usepackage{amssymb}
\usepackage{mathrsfs}
%\usepackage{minted}
\usepackage{bm}
\usepackage{amsmath}
\usepackage{enumitem}
\usepackage[english]{varioref}
\usepackage[english]{babel}
\usepackage{lipsum}
\usepackage{fancyhdr}
\pagestyle{fancy} 
\usepackage{float}
\usepackage{empheq}
\usepackage[framemethod=tikz]{mdframed}
\usepackage{epstopdf}
\numberwithin{equation}{section}
\usepackage{eso-pic}
\usepackage{calc}
\usepackage{nccmath}
\usepackage{caption}
\usepackage{subcaption}
\usepackage{gensymb}
\usepackage{amsfonts,amsthm,epsfig,epstopdf,titling,url,array}
\usepackage{siunitx}
\sisetup{input-digits = 0123456789\pi}
\usepackage[symbol]{footmisc}
\usepackage{xcolor}
\usepackage{multicol}
\usepackage{boondox-cal}
\DeclareSIUnit\atm{atm}
\setcounter{secnumdepth}{3}
\setcounter{tocdepth}{3}
\usepackage{booktabs}
\usepackage{blindtext}
\usepackage{changepage}

% \usepackage{draftwatermark}
% \SetWatermarkText{DRAFT}
% \SetWatermarkScale{5}

\DeclareSIUnit\atm{atm}

\pagestyle{fancy} 
\fancypagestyle{firstpage}{
	\rhead{
		%	\begin{picture}(0,0) 
			%			\put(-30,0){\includegraphics[width=1cm]{figures/MCI_4C_bw.eps}} 
			%	\end{picture}
	}
}
\fancyhead[L]{\slshape\nouppercase{\leftmark}}
\chead{}
\rhead{
	%	\begin{picture}(0,0) 
		%		\put(-30,0){\includegraphics[width=1cm]{figures/MCI_4C_bw.eps}} 
		%	\end{picture}
}
\lfoot{\textit{}}
\cfoot{-\ \thepage\ -}
\rfoot{\textit{}}

\DeclareMathOperator{\rank}{rank}
\DeclareMathOperator{\atantwo}{atan2}
\DeclareMathOperator{\spn}{span}

\renewcommand{\headrulewidth}{0.4pt}
\renewcommand{\footrulewidth}{0.4pt}
\newcommand{\abs}[1]{\left|#1\right|}
\definecolor{mycolor1}{rgb}{0.97, 0.97, 0.97}
\definecolor{mycolor2}{rgb}{0.97, 0.97, 0.97}
\definecolor{tableShade}{gray}{0.9}
\newcommand{\sign}{\text{sign}}
\newcommand{\centered}[1]{\begin{tabular}{@{}l@{}} #1 \end{tabular}}
\theoremstyle{it}
\newtheorem{defn}{Definition}[chapter]
\newtheorem{assumption}{Assumption}[chapter]
\newtheorem{thm}{Theorem}[chapter]
\newtheorem{lemma}{Lemma}[chapter]
\newtheorem{corollary}{Corollary}[chapter]
%\newtheorem{defn}{Definition}[section]
%\newtheorem{assumption}{Assumption}[section]
%\newtheorem{thm}{Theorem}[section]
%\newtheorem{lemma}{Lemma}[section]
%\newtheorem{corollary}{Corollary}[section]
\theoremstyle{definition}
%\theoremstyle{it}
\newtheorem{example}{Example}[section]

\newenvironment{myitemize_1}
{ \begin{itemize}[topsep=0pt]
		\setlength{\topsep}{2pt}		
		\setlength{\itemsep}{2pt}
		\setlength{\parskip}{2pt}
		\setlength{\parsep}{2pt}     }
	{ \end{itemize}                  }



\newmdenv[innerlinewidth=0.5pt, roundcorner=4pt,backgroundcolor=mycolor2, 
linecolor=mycolor1,innerleftmargin=6pt,
innerrightmargin=6pt,innertopmargin=6pt,innerbottommargin=6pt]{mybox}

\title{\textbf{ 
		\begin{LARGE}
			DriverBoard Test Evaluation
		\end{LARGE} \\[24pt]
		\begin{Large}
			Preliminary wishes
	\end{Large}}
}
\author{Davide Bagnara, Gernot Landskron}

\begin{document}
	\begin{onehalfspace}
		\thispagestyle{firstpage}
		\begin{mybox}
			\maketitle
			\vspace{110mm}
		\end{mybox}
		\newpage
		\tableofcontents
		\listoffigures	
		\listoftables
		\newpage
		
		\chapter{Introduction}	
		In the following chapter a brief introduction to the Flying Basket firmware implementation is reported. Figure~\ref{electrical_circuit_3} shows the fundamental system architecture. From \textbf{functionality} point of view it is possible to select the following main structures:
		\begin{itemize}
			\item[--] power stage (inverter, dclink, motor);
			\item[--] auxiliary power supply, gate drivers, high side voltage and current measure systems;
			\item[--] low side voltage and current measure conditioning and hardware protections;
			\item[--] $\mu$-processor based control architecture;
			\item[--] field bus communication (interface to external environment).
		\end{itemize} 
		Before to proceed forward a short list of symbols and acronyms used in this document is reported, as follows:
		\begin{itemize}
			\item[--] 	OCP: over-current protection.
			\item[--]	DMA: direct memory access.
			\item[--]   ADC: analog to digital converter.	
			\item[--] 	DAC: digital to analog converter.
			\item[--] 	GPIO: general purpose I/O.	
			\item[--] 	SVPWM: space vector pulse-width modulation.
			\item[--] 	PMSM: permanent magnet synchronous motor.	
			\item[--] 	MTPA: max torque per ampere.
			\item[--] 	TIMx: timers available into microcontroller.
			\item[--] 	HAL: hardware abstraction layer.
			\item[--] 	n.a.: not applicable.
		\end{itemize}

		
		\clearpage
		\begin{thebibliography}{99}
			\bibitem{volke} 
			A. Volke, M. Hornkamp, \emph{IGBT Modules. Technologies, Driver and Application}. Infineon 2011.			
		\end{thebibliography}
	\end{onehalfspace}
\end{document} 