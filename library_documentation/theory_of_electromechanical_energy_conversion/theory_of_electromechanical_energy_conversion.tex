	%\documentclass[11pt,a4paper]{article}
\documentclass[11pt,a4paper]{scrartcl}
%\documentclass[11pt,a4paper]{scrartcl}
\usepackage[british,UKenglish,USenglish,english,american]{babel}
%\usepackage[a4paper, total={16cm, 23cm}]{geometry}
\usepackage[tmargin=1.25in,bmargin=1.25in,lmargin=1in,rmargin=1in]{geometry}
\usepackage{tikz}
\usepackage{graphicx}

\usepackage{chemmacros}
\usepackage{chemfig}
%\usepackage{ghsystem}
\usechemmodule{redox}
%\usepackage{chemnum}
%\usepackage{bohr}
%\usepackage{elements}
%\usepackage{endiagram}
%\usepackage{modiagram}
%\usepackage{chemgreek}
\usepackage{mhchem}


\usepackage{makeidx}
\usepackage{epstopdf}

\usepackage{amssymb}
\usepackage{mathrsfs}
%\usepackage{minted}
\usepackage{amsmath}
\usepackage{enumitem}
\usepackage[english]{varioref}
\usepackage[english]{babel}
\usepackage{lipsum}
\usepackage{fancyhdr}
\pagestyle{fancy} 
\usepackage{float}
\usepackage{empheq}
\usepackage[framemethod=tikz]{mdframed}
\usepackage{epstopdf}
\numberwithin{equation}{section}
\usepackage{eso-pic}
\usepackage{calc}
\usepackage{nccmath}
\usepackage{caption}
\usepackage{subcaption}
\usepackage{gensymb}
\usepackage{amsfonts,amsthm,epsfig,epstopdf,titling,url,array}
\usepackage{siunitx}
\usepackage{xcolor}
\usepackage{multicol}
\usepackage{boondox-cal}
\DeclareSIUnit\atm{atm}

\fancypagestyle{firstpage}{
	\rhead{\begin{picture}(0,0) \put(-30,0){\includegraphics[width=1cm]{figures/MCI_4C_bw.eps}} \end{picture}}
}
\lhead{\textit{Theory of Electromechanical Energy Conversion}}
\chead{}
\rhead{\begin{picture}(0,0) \put(-30,0){\includegraphics[width=1cm]{figures/MCI_4C_bw.eps}} \end{picture}}
\lfoot{\textit{Spring 2021}}
\cfoot{-\ \thepage\ -}
\rfoot{\textit{}}
\renewcommand{\headrulewidth}{0.4pt}
\renewcommand{\footrulewidth}{0.4pt}
\newcommand{\abs}[1]{\left|#1\right|}
\definecolor{mycolor1}{rgb}{0.95, 0.95, 0.95}
\definecolor{mycolor2}{rgb}{0.95, 0.95, 0.95}
\newcommand{\sign}{\text{sign}}
\theoremstyle{it}
\newtheorem{defn}{Definition}[section]
\newtheorem{thm}{Theorem}[section]
\newtheorem{lemma}{Lemma}[section]
\theoremstyle{definition}
\theoremstyle{it}
\newtheorem{example}{Example}[section]


\newmdenv[innerlinewidth=0.5pt, roundcorner=4pt,backgroundcolor=mycolor2, linecolor=mycolor1,innerleftmargin=6pt,
innerrightmargin=6pt,innertopmargin=6pt,innerbottommargin=6pt]{mybox}

\title{Theory of Electromechanical Energy Conversion}
\author{Davide Bagnara}

\begin{document}
	\maketitle
	\thispagestyle{firstpage}
	\tableofcontents
	\listoffigures	
	\listoftables		

\section{Introduction.} 
This lecture covers the topic regarding the adaptive control. In particular we will focus on \textit{Model Reference Adaptive Control} in which a reference model is taken as benchmark and a defined set of parameters are adapted in order to keep the answer of the system close to the reference one. Most of the material presented here is taken from \cite{p18}, \cite{p25} and \cite{p23}.


\begin{thebibliography}{99}
	\bibitem[\textbf{A. Mehrle, 2020}]{p1} Andreas Mehrle - \textit{Advanced Control Engineering}. MCI 2020.
	\bibitem[\textbf{G. Tao 2003}]{p18} G. Tao - \textit{Adaptive Control Design and Analysis}. Wiley-Interscience 2003.
	\bibitem[\textbf{M. Sunwoo 1991}]{p19} M. Sunwoo, Ka C. Cheok, N. J. Huang - \textit{Model Reference Adaptive Control for Vehicle Active Suspension Systems}. IEEE Transaction on industrial electronics 1991.		
	\bibitem[\textbf{H. K. Khalil 2002}]{p22} H. K. Khalil - \textit{Nonlinear Systems}. Third edition, Prentice Hall 2002.
	\bibitem[\textbf{J.J.E. Slotine 1991}]{p23} J.J.E. Slotine, W. Li - \textit{Applied Nonlinear Control}. Prentice Hall 1991.
	\bibitem[\textbf{D.R. Merkin 1996}]{p24} D.R. Merkin - \textit{Introduction to the Theory of Stability}. Springer Hall 1993.
	\bibitem[\textbf{P.A. Ioannou 2012}]{p25} P.A. Ioannou, J. Sun - \textit{Robust Adaptive Control}. Dover 2012.
\end{thebibliography}

	
%\end{enumerate}

\end{document} 